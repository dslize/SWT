Ohne ein zu überprüfendes Programm, muss man die Testfälle aus der Spezifikation ableiten.
Dazu kann man den Blackbox Test einsetzen, da man keine Kenntnis hat über die Struktur des Programms, wohl aber über die gewünschte Funktionalität.
(Siehe Aufgabenstellung.) Man kann das Verhalten der Funktion durch folgende Mengen testen:
\begin{itemize}
\item Gerade Zahlen \{2, 4, 6, …\}
\item Ungerade Zahlen \{1, 3, 5, …\}
\item Mischung von zufälligen natürlichen Zahlen \{1, 2, 42, …\}
\item Wir überprüfen 0 als Spezialfall, da ohne weitere Definition nicht klar ist, ob hier 0 in den natürlichen Zahlen liegt
\end{itemize}
Die Funktion erwartet eine natürliche Zahl, dabei sollten wir aber vor dem Überprüfen,
ob die Zahl gerade ist, erstmal feststellen, ob die Eingabe eine gültige Eingabe ist.
Was passiert bei z.B. Zeichenstrings, negativen Zahlen, Zahlenwerte mit Nachkommastellen usw.\\ \ \\
Die konkreten Werte die man zum Testen verwenden kann sind: \texttt{-101, -100, 0, 64, 65, -0.5, 0.6}
und ein beliebiges String das man vielleicht mit den Tasten des Taschenrechners eingeben kann z.B. „\texttt{(“} ;