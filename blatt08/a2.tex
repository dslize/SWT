\aufgabenteil{a)}
Eine Komponente im Software Engineering ist eine abgeschlossene Einheit eines Software-Systems, die
meist eine logisch getrennte Aufgabe erfüllt und die dazu benötigten Inhalte zusammenfasst. Komponenten
zeichnen sich durch eine hohe Kohäsion nach innen (Zusammengehörigkeit der Bestandteile der Komponente) und lose Kopplung nach außen (geringe Abhängigkeit von anderen Komponenten) aus. Komponenten
existieren in einem Kontext: sie werden von anderen Komponenten verwendet und verwenden ihrerseits
auch andere Komponenten. Dazu ist es nötig, Schnittstellen explizit zu definieren.\par
Neben der Schaffung einer logischen Struktur bringen Komponenten den Vorteil der Wiederverwendung
und Austauschbarkeit. Ein Softwaresystem könnte beispielsweise auf einen Kunden zugeschnitten werden,
indem eine bestimmte Komponente ausgetauscht wird.

\aufgabenteil{b)}
Software Komponenten\dots
\begin{itemize}
\item sind von ihrer Umgebung möglichst unabhängig ($\Rightarrow$ Austauschbarkeit)
\item haben explizite Schnittstellen, verringern dadurch Kopplung
\item fördern Wiederverwendbarkeit und Qualität (``best in class'', Produktivitätssteigerung, geringerer Schulungsaufwand etc.)
\item können unabhängig versionsverwaltet werden
\end{itemize}

\aufgabenteil{c)}
Mit Glue Code wird derjenige Programmcode bezeichnet, der die Komponenten untereinander verbindet.
Dies kann auf verschiedenen Arten erfolgen (z.B. durch Scripting-Sprachen) und ist recht einfach auszutauschen.