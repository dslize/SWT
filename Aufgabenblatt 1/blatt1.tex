\documentclass[a4paper,graphics,11pt]{article}

\usepackage[utf8]{inputenc}
\usepackage[T1]{fontenc}
\usepackage{lmodern}
\usepackage[ngerman]{babel}
\usepackage{amsmath, tabu}
\usepackage{amsthm}
\usepackage{amssymb}
\usepackage{algorithm}
\usepackage{algpseudocode}
\usepackage{mathtools}
\usepackage{setspace}
\usepackage{graphicx,color,curves,epsf,float,rotating}

\floatname{algorithm}{Algorithmus}

\newcommand\norm[1]{\left\lVert#1\right\rVert}
\newcommand\abs[1]{\left\vert#1\right\vert}

\newcommand\aufgabe[1]{\subsection*{Aufgabe #1}}
\newcommand\aufgabenteil[1]{\subsubsection*{#1}}



\pagestyle{empty}
\begin{document}
\noindent WS 2016/17        \hfill Simon Kaiser, 354692 \\
\null                                     \hfill Philipp Hochmann, 356148 \\
\null                                     \hfill Felix Kiunke, 357322 \\
\null                                     \hfill Daniel Schleiz, 356092 \\

\begin{center}
\Large \textsc{Softwaretechnik} \\   % Fach
\large Aufgabenblatt 1                        % Nummer das Blattes, nicht vergessen zu ändern!
\end{center}
\begin{center}
\rule[0.5ex]{\textwidth}{0.6pt}\vspace*{-\baselineskip}\vspace{3.2pt}
\rule[0.5ex]{\textwidth}{1.6pt}\\
\end{center}


%%%%%%%%%%%%%%%%%%%%%%%%%%%%%%%%%%%%%%
%
%   Ab hier kommt der Text
%   Neue Aufgabe mit \aufgabe{}
%   Aufgabenteil mit \aufgabenteil{}
% 
%%%%%%%%%%%%%%%%%%%%%%%%%%%%%%%%%%%%%%

\aufgabe{1.1}

\textbf{Analyse}
\begin{itemize}
\item Qualitätssicherung des Pflichtenheftes durchführen\\
Das Pflichtenheft entsteht in der Analyse-Phase und dient als ein wichtiges Kommunikationsmittel zwischen dem Kunden und dem Projektleiter. Dessen Qualität sollte
stets gesichert sein.
\item Gesetzliche Rahmenbedingungen prüfen\\
Es ist sinnvoll, schon in der Analyse-Phase die gesetzlichen Rahmenbedingungen zu prüfen, um festzustellen, ob die Wünsche des Kunden überhaupt rechtlich gesehen
umsetzbar sind und im Zweifelsfall nach Alternativen zu suchen.
\item Entwicklerteam zusammenstellen \\
Während der Analyse sollte klar werden, in welche Richtung die Anwendung gehen soll und bei Bedarf kann man sich schon um Fachkräfte für absehbare Spezialgebiete
kümmern. Zudem ist man bei einer so frühen Planung des Teams recht flexibel und kann im Notfall später noch Verstärkung suchen.
\item Termine und Kosten des Projektes planen \\
Es sollten jetzt schon die Termine und Kosten für das Projekt geplant werden, damit das Projekt nicht außer Kontrolle gerät. 
\item Vorhandene Altlasten des Kunden analysieren \\
Anhand von Software Altlasten des Kunden entsteht ein genaueres Bild, was der Kunde tatsächlich haben oder eben nicht haben möchte, was wiederum in die Analyse
einfließen sollte.
\item Kunden eine Rechnung stellen \\
Da man die Kosten bereits anhand der Anforderungen prognostizieren kann, ist ein erster Kostenanschlag in der Analyse-Phase ebenfalls vernünftig. Zudem erspart man sich
so spätere Streitigkeiten mit dem Kunden und kann einen Kompromiss finden. Die letztendliche Rechnung ist aber während der Integration zu stellen.
\item Nach bereits vorhandenen, wiederverwendbaren Software-Bibliotheken suchen\\
\end{itemize} \ \\ \ \\
\textbf{Entwurf}
\begin{itemize}
\item Konzept und Prototyp einer Benutzeroberfläche erstellen\\
Erste Konzepte und der Prototyp einer Benutzerfläche sollten beim Entwurf entwickelt werden, da diese für den Kunden eine der wichtigsten Sachen darstellt und man
bei Nicht-Gefallen darauf reagieren kann. Außerdem fallen bei der Erstellung des Konzeptes der Benutzeroberfläche noch eventuell weitere Funktionalitäten auf, welche in 
den Entwurf mit eingebunden werden sollten.
\item Datenstrukturen festlegen\\
Während man sich bei der Analyse bereits Gedanken über mögliche Datenstrukturen machen kann, sollte man sich beim Entwurf auf geeignete und effiziente Datenstrukturen
festlegen.
\item Schnittstellen von Programmmodulen definieren\\
Im Entwurf sollten Programmmodule des Produkts ersichtlich werden. Mittels Definitionen von Schnittstellen lassen sich diese dann sinnvoll vernetzen.
\item Strukturmodell des gesamten Softwaresystems entwerfen \\
Der zentrale Aspekt des Entwurfes ist es, sich ein Strukturmodell des Projekts zu überlegen.
\item Performance-Prognose des Softwaresystems erstellen\\
Da beim Entwurf ein Gesamteindruck der Software entsteht, kann man eine erste Prognose der Performance des Systems erstellen und diese in der Implementierungsphase
sinnvoll nutzen, um sich gewisse Ziele in Hinsicht auf Performance zu setzen.
\end{itemize} \ \\ \ \\
\textbf{Implementierung}
\begin{itemize}
\item Code eines Programmmoduls debuggen\\
Während der Implementierung bringt ein stumpfes Programmieren ohne Tests natürlich nichts; den Code eines Moduls zu debuggen ist deswegen elementar.
\item Leistung der Entwickler bewerten und belohnen\\
Schon im Implementierungsprozess sind bei Benutzung einer geeigneten Verwaltungssoftware die individuellen Leistungen der Entwickler ersichtlich. Eine Belohnung guter
Leistungen kann dabei die Moral erhöhen und so die Qualität der Implementierung erhöhen.
\item Nach bereits vorhandenen, wiederverwendbaren Software-Bibliotheken suchen\\
Beim Programmieren fällt vielleicht auf, dass für gewisse Funktionalitäten bereits gute Bibliotheken existieren, wodurch sich Zeit sparen lässt und der Code meist weniger
fehleranfällig ist.
\item Programmcode kommentieren \\
Während der Implementierung sollte der Code natürlich stets kommentiert werden, damit er später sauber gewartet oder erweitert werden kann.
\end{itemize} \ \\ \ \\
\textbf{Test-Integration}
\begin{itemize}
\item Benutzer der Software schulen\\
Bei der Integration sollte der Kunde beziehungsweise der letztendliche Nutzer natürlich zur Nutzung geschult werden, da sonst vielleicht einige Features falsch
verstanden, falsch genutzt oder als Fehler interpretiert werden können.
\item Zwei Subsysteme verbinden und testen\\
Zwar sollten beim Implementieren zumindest rudimentäre Tests über die Kompatibilität von Subsystemen durchgeführt werden, ausführliche Tests finden aber in dieser
Phase statt. 
\item Kunden eine Rechnung stellen \\
Die letztendliche Rechnung kann nun gestellt werden, da am Ende dieser Phase einerseits das meiste der Software steht und andererseits der kommende Wartungsaufwand ersichtlich ist.
\item Test-Eingabedaten für ein Programmmodul ermitteln\\
Zum ausgiebigen Testen ist es notwendig, sinnvolle und realitätsnahe Testdaten für ein Programmmodul zu ermitteln.
\end{itemize} \ \\ \ \\
\textbf{Wartung}
\begin{itemize}
\item Software an neue Umgebung anpassen\\
Falls sich in der Umgebung des Kunden fundamentale Sachen ändern, muss die Software gewartet und abgeändert werden, um den Anforderungen zu entsprechen.
\item Dokumentation des Projektablaufes bewerten und archivieren\\
Das Projekt sollte nun bis auf die Wartung abgeschlossen sein, weshalb nun eine Bewertung und Archivierung der Dokumentation erfolgen kann.
\end{itemize} \ \\ \ \\
Eine solche starre Einteilung von Aktivitäten in Phasen wie im Wasserfallmodell ist problematisch, da dieses System sehr statisch ist und manche Aktivitäten wie 
zum Beispiel die Qualitätssicherung stets stattfinden sollten, um später größere Probleme zu vermeiden. Tritt beispielsweise in einer frühen Phase ein großer Fehler auf,
welcher im Wasserfallmodell erst in einer viel späteren Phase durch eine Aktivität aufgedeckt wird, so wird unter Umständen geleistete Arbeit obsolet und man muss
zu einer früheren Phase zurückspringen.
Außerdem ist im Wasserfallmodell die Kommunikation mit dem Kunden in der Entwurfs- und der Implementierungsphase eher nicht vorgesehen, was aber prinzipiell
eine schlechte Idee ist, da der Kunden in den Entwicklungsprozess involviert sein sollte, da dieser letzten Endes das Produkt akzeptieren soll.

\aufgabe{1.2}

Ein zentraler Gesichtspunkt der SCRUM-Entwicklung ist das Sprint-basierte Arbeitsverhalten, wodurch man zeitlich begrenzte Phase für bestimmte Aufgaben hat und 
dann die Entwicklung für den nächsten Sprint erneut organisiert wird. Man arbeitet also vor allem nicht ergebnisorientiert, sondern zeitorientiert. \\
Zudem weist SCRUM einer inkrementelle Arbeitsweise auf, welche auf den Sprint-Backlogs beruht. Es werden für einen Sprint von der Gesamtheit der zu erledigenden
Aufgaben ein gewisser Teil herausgenommen und zu Sprint-Backlogs organisiert, welche dann abgearbeitet werden können. Ist der Sprint vorbei, so werden die
Backlogs wieder zurückgesteckt und es geschieht eine neue Organisation in in Backlogs aus der Gesamtheit der Aufgaben für den nächsten Sprint. \\
Zuletzt lassen sich durch diese Art der Softwareentwicklung die geleisteten Ergebnisse transparent darstellen, der Dialog zum Kunden ist also in sämtlichen
Phasen sehr wichtig. \par
Die wohl wichtigste Eigenschaft des extreme Programming ist die inkrementelle, evolutionäre Entwicklung eines permanent lauffähigen Codes. In diesem sind dann
die Ergebnisse stets sichtbar. \\
Daraus folgt eine zweite grundlegende, ein wenig kuriose Eigenschaft: Der Code selber und die ständig darauf ausgeführten Tests sind dann de facto das Entwurfsdokument
und die Dokumentation, welche im klassischen Sinne so nicht existieren. Insbesondere das disziplinierte Testen des Codes und die daraus entstehenden Dokumente sorgen
für eine Qualitätssicherung. \\
Nichtsdestotrotz sollte beachtet werden, dass mit dieser Art der Entwicklung in gewissem Maße Chaos entstehen kann bei zu großen Teams, es ist also empfehlenswert
eher kleinere Projekte damit zu realisieren.

\aufgabe{1.3}
\textbf{Funktionale Anforderungen}\\
\begin{itemize}
\item Die App muss sich mit dem Stundenplan des vorhandenen Management Systems der Uni synchronisieren können\\
\item Es sollte eine integrierte Accountverwaltung implementiert werden\\
\item Als Identifikation soll ein verbinden mit social-network Profilen ermöglicht werden\\
\item Eine Geolocationfunktion zum Finden von Übungspartnern \\
\item Nachrichten verschicken und empfangen von anderen Personen\\
\item Ein Live-Chatsystem mit Gruppenchats \\
\end{itemize}

\noindent \textbf{Funktionale Anforderungen}\\
\begin{itemize}
\item Der Geolocation Service soll nur im Umkreis von 5km nach Übungspartnern suchen\\
\item Die Anwendung soll in C++ geschrieben werden\\
\item Die Erstellung eines Accounts soll schnell und unkompliziert sein und nicht zu viele Informationen abfragen, mehr optionale Felder\\
\item Die grafische Oberfläche soll der des Management Systems der Universität nachempfunden sein, damit der Umstieg nicht allzu schwer ist\\
\item Die Anwendung soll gut skalierbar sein und mit vielen angemeldeten Usern zurechtkommen\\
\end{itemize}


\end{document}
% Nummer des Blattes angepasst?