\aufgabenteil{a)}
Ohne Betrachtung des Programmcodes wäre das Black-Box Verfahren eine geeignete Weise zum Testen. In dem Rahmen möchte man sich möglichst
geschickt Eingabedaten überlegen, womit man die Methode testen könnte.\\
Dazu ist es sinnvoll, sich Äquivalenzklassen für die Eingabedaten zu entwickeln, um mit so wenig wie möglich Eingaben die Methode so umfangreich
wie es geht zu testen. Die Äquivalenzklassen von den Eingabedaten für diese Methode könnten wie folgt aussehen:
\begin{itemize}
    \item x < 0
    \item x > 0
    \item x = 0
\end{itemize}
Insbesondere sollte die 0 geprüft werden, da unter Umständen in der Methode Divisionen vorkommen könnten, wobei 0 als Nenner problematisch wäre.
Außerdem sollte man noch explizit Integer.MAX\_VALUE und Integer.MIN\_VALUE testen, da die Methode arithmetische Operationen verwenden könnte, welche
Überläufe ausgelöst werden könnten, was das Ergebnis verfälschen würde. Da außerdem auch null in die Methode übergeben werden könnte, sollte geprüft
werden, wie die Methode damit umgeht.

\aufgabenteil{b)}

\begin{itemize}
    \item x < 0: Eingabemenge: \{ Integer.MIN\_VALUE, -49783, -14, -1\}. Neben dem oben genannten Randfall Integer.MIN\_VALUE sollte auch noch -1 als Randfall
        betrachtet werden, da eventuell arithmetische Operationen durchgeführt werden, wodurch die Eingabe in den positiven Bereich rutscht. Außerdem sollen
        mit -14 und -49783 noch ein zufälliger ``kleiner'' und ``großer'' Wert getestet werden, da nur Randfallbetrachtungen nicht ausreichen.
    \item x > 0: Eingabemenge: \{ 1, 15, 59285, Integer.MAX\_VALUE\}. Neben dem oben genannten Randfall Integer.MAX\_VALUE sollte auch noch 1 als Randfall
        betrachtet werden, da eventuell arithmetische Operationen durchgeführt werden, wodurch die Eingabe in den negativen Bereich rutscht. Außerdem sollen
        mit 15 und 59285 noch ein zufälliger ``kleiner'' und ``großer'' Wert getestet werden, da nur Randfallbetrachtungen nicht ausreichen.
    \item x = 0: Eingabemenge: \{ 0 \}.
\end{itemize}
Zudem sollte auch noch null als Eingabe getestet werden.