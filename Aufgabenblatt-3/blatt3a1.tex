\begin{figure}[h!]
\begin{tikzpicture}
\begin{umlseqdiag}
	\umlobject[class=Kunde]{k}
	\umlobject[class=Mitarbeiter,x=6]{m}
%	\umlobject[class=Pommes,x=11]{p}

	\begin{umlcall}[op=fragtNach(Pommes),dt=5]{k}{m}
		\begin{umlcall}[op=rückfrage(Mayo),return=ja,dt=5]{m}{k}
		\end{umlcall}
		
		\begin{umlcreatecall}[class=Pommes,x=11,op=vorbereiten()]{m}{p}
		\end{umlcreatecall}

		\begin{umlcallself}[op=warten(),dt=6]{k}
		\end{umlcallself}

		\begin{umlcall}[type=return,op=p,dt=6]{p}{m}

			\begin{umlcall}[op={bestellungLiefern(p, mayo)}]{m}{k}
			\end{umlcall}

			\begin{umlcall}[op=essen(schnell)]{k}{p}
			\end{umlcall}

		\end{umlcall}

		\umlsdnode[dt=1,name=end-p]{p}

	\end{umlcall}

	\tikzset{cross/.style={cross out, draw=black, fill=none, minimum size=2*(#1-\pgflinewidth), inner sep=0pt, outer sep=0pt}, cross/.default={2pt}}
	\draw[line width=0.5mm] 
		($ (end-p) - (0.2,0.5) $) -- ++(.4,.4)
		($ (end-p) - (-.2,0.5) $) -- ++(-.4,.4);

\end{umlseqdiag}
\end{tikzpicture}
\end{figure}
