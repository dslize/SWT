\documentclass[a4paper,graphics,11pt]{article}

\usepackage[utf8]{inputenc}
\usepackage[T1]{fontenc}
\usepackage{lmodern}
\usepackage[ngerman]{babel}
\usepackage{amsmath, tabu}
\usepackage{amsthm}
\usepackage{amssymb}
\usepackage{algorithm}
\usepackage{algpseudocode}
\usepackage{mathtools}
\usepackage{setspace}
\usepackage{graphicx,color,curves,epsf,float,rotating}
\usepackage{tikz}
\usepackage{../tikz-uml}

\floatname{algorithm}{Algorithmus}

\newcommand\norm[1]{\left\lVert#1\right\rVert}
\newcommand\abs[1]{\left\vert#1\right\vert}

\newcommand\aufgabe[1]{\subsection*{Aufgabe #1}}
\newcommand\aufgabenteil[1]{\subsubsection*{#1}}



\pagestyle{empty}
\begin{document}
\noindent WS 2016/17        \hfill Simon Kaiser, 354692 \\
\null                                     \hfill Philipp Hochmann, 356148 \\
\null                                     \hfill Felix Kiunke, 357322 \\
\null                                     \hfill Giacomo Klingen, 356778 \\
\null                                     \hfill Daniel Schleiz, 356092 \\

\begin{center}
\Large \textsc{Softwaretechnik} \\   % Fach
\large Aufgabenblatt 5                        % Nummer das Blattes, nicht vergessen zu ändern!
\end{center}
\begin{center}
\rule[0.5ex]{\textwidth}{0.6pt}\vspace*{-\baselineskip}\vspace{3.2pt}
\rule[0.5ex]{\textwidth}{1.6pt}\\
\end{center}


%%%%%%%%%%%%%%%%%%%%%%%%%%%%%%%%%%%%%%
%
%   Ab hier kommt der Text
%   Neue Aufgabe mit \aufgabe{}
%   Aufgabenteil mit \aufgabenteil{}
% 
%%%%%%%%%%%%%%%%%%%%%%%%%%%%%%%%%%%%%%

\aufgabe{5.1}
Man trennt die Software-Produktlinie in zwei Bereiche, welche abhängig voneinander entwickelt werden.\\

Die erste Phase wird als \textit{Domänen Engineering} bezeichnet. Man erfasst alle relevanten Konzepte der Domäne, aus der 
man die Realisierung einer wiederverwendbaren Plattform produziert. (Man muss sowohl die Gemeinsamkeiten als auch die 
Unterschiede der einzelnen Produkte einbauen.) Durch die Wiederverwendung der Komponenten spart man bei den
Produktionskosten und trägt zu einer besseren Qualität der Komponenten bei.\\

Um die Endprodukte zugeschnitten auf die Bedürfnisse der einzelnen Kunden zu entwickeln, 
werden in dem zweiten Bereich, das \textit{Applikation Engineering}, die wiederverwendbaren Komponenten der Plattform 
ausgewählt und miteinander integriert. Dabei wird mit dem Domänen Engineering insofern zusammengearbeitet, 
dass wiederverwendbare Komponenten bestimmt werden.

\aufgabe{5.2}

\begin{figure}[H]
  \centering
  \input{swt52.pdf_t}
\end{figure}

\aufgabe{5.3}

\begin{table}[h]
\centering
\begin{tabular}{c|c}
Aufgabenteil & Antwort \\ \hline
a) & ja   \\
b) & nein \\
c) & nein \\
d) & ja   \\
e) & nein \\
f) & nein \\
g) & ja   \\
h) & ja   \\
i) & UML: nein, Java: ja
\end{tabular}
\end{table}

\end{document}
% Nummer des Blattes angepasst?
