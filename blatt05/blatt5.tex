\documentclass[a4paper,graphics,11pt]{article}

\usepackage[utf8]{inputenc}
\usepackage[T1]{fontenc}
\usepackage{lmodern}
\usepackage[ngerman]{babel}
\usepackage{amsmath, tabu}
\usepackage{amsthm}
\usepackage{amssymb}
\usepackage{algorithm}
\usepackage{algpseudocode}
\usepackage{mathtools}
\usepackage{setspace}
\usepackage{graphicx,color,curves,epsf,float,rotating}

\floatname{algorithm}{Algorithmus}

\newcommand\norm[1]{\left\lVert#1\right\rVert}
\newcommand\abs[1]{\left\vert#1\right\vert}

\newcommand\aufgabe[1]{\subsection*{Aufgabe #1}}
\newcommand\aufgabenteil[1]{\subsubsection*{#1}}



\pagestyle{empty}
\begin{document}
\noindent WS 2016/17        \hfill Simon Kaiser, 354692 \\
\null                                     \hfill Philipp Hochmann, 356148 \\
\null                                     \hfill Felix Kiunke, 357322 \\
\null                                     \hfill Giacomo Klingen, 356778 \\
\null                                     \hfill Daniel Schleiz, 356092 \\

\begin{center}
\Large \textsc{Softwaretechnik} \\   % Fach
\large Aufgabenblatt 5                        % Nummer das Blattes, nicht vergessen zu ändern!
\end{center}
\begin{center}
\rule[0.5ex]{\textwidth}{0.6pt}\vspace*{-\baselineskip}\vspace{3.2pt}
\rule[0.5ex]{\textwidth}{1.6pt}\\
\end{center}


%%%%%%%%%%%%%%%%%%%%%%%%%%%%%%%%%%%%%%
%
%   Ab hier kommt der Text
%   Neue Aufgabe mit \aufgabe{}
%   Aufgabenteil mit \aufgabenteil{}
% 
%%%%%%%%%%%%%%%%%%%%%%%%%%%%%%%%%%%%%%

\aufgabe{5.1}

TODO

\aufgabe{5.2}

TODO

\aufgabe{5.3}

\begin{table}[h]
\centering
\begin{tabular}{c|c}
Aufgabenteil & Antwort \\ \hline
a) & ja   \\
b) & nein \\
c) & nein \\
d) & ja   \\
e) & nein \\
f) & nein \\
g) & ja   \\
h) & ja   \\
i) & UML: nein, Java: ja
\end{tabular}
\end{table}

\end{document}
% Nummer des Blattes angepasst?